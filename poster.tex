\documentclass[25pt,a1paper]{tikzposter}

%% Tikzposter is highly customizable: please see
%% https://bitbucket.org/surmann/tikzposter/downloads/styleguide.pdf

%% Available themes: see also
%% https://bitbucket.org/surmann/tikzposter/downloads/themes.pdf
% \usetheme{Default}
% \usetheme{Rays}
% \usetheme{Basic}
% \usetheme{Simple}
\usetheme{Envelope}
% \usetheme{Wave}
% \usetheme{Board}
% \usetheme{Autumn}
% \usetheme{Desert}

%% Further changes to the title etc is possible
% \usetitlestyle{Default}
% \usetitlestyle{Basic}
% \usetitlestyle{Empty}
% \usetitlestyle{Filled}
% \usetitlestyle{Envelope}
% \usetitlestyle{Wave}
% \usetitlestyle{verticalShading}

\usepackage{fontspec}
\setmainfont{FreeSerif}
\setsansfont{FreeSans}

%\author{Me Me Me}
\title{口腔癌前病變的精準預防}
%\institute{ACME Institute}
%% Optional title graphic
\titlegraphic{\includegraphics[width=27cm]{logo_TMUH.jpg}}
%% Uncomment to switch off tikzposter footer
% \tikzposterlatexaffectionproofoff


%\usepackage{CJKutf8} % by pdfLaTeX, not LuaLaTeX
% *** https://www.math.sinica.edu.tw/www/tex/default14.jsp
\usepackage{xeCJK} % for Chinese, compiling by XeLaTex

\usepackage{fontspec} %設定字體
% Fandol font (the default)  not shown "內"
\setCJKmainfont{AR PL UMing TW MBE} % AR PL UMing TW MBE or "UKai" https://www.overleaf.com/learn/latex/Questions/Which_OTF_or_TTF_fonts_are_supported_via_fontspec%3F#Chinese
%BiauKai} %標楷體 from macOS %設定中文為系統上的字型,而英文不去更動,使用原TeX字型

\XeTeXlinebreaklocale "zh"
\XeTeXlinebreakskip = 0pt plus 1pt %這兩行一定要加,中文才能自動換行

\usepackage{outlines}

\begin{document}
\maketitle

%
\note[rotate=8, connection, width = 10cm,
%roundedcorners=15, 
targetoffsetx=+15cm, %-0.01\textwidth,
targetoffsety=-35cm
]{\includegraphics[width=1.0\linewidth]{TMUH_Wu_QRcode.png}}


\block{計畫名稱:開發高危險口腔癌前病變的精準診斷治療策略---聚焦口腔疣狀增生}{

\begin{outline}
    \1 計畫目的:口腔疣狀增生是一種外形為疣狀(或乳頭狀)的癌前病變;這種病變日後有較高的風險,會轉變為口腔癌。我們的目標是找出有效的生物標記,期望能無需頻繁切片,就能早期預警並早期治療。
    \1  受試者應配合之事項:預計收集剩餘切片檢體(surgical specimen),加上血液、唾液,以進行數位病理影像分析、血清與唾液的基因體分析。並接受每三個月門診定期追蹤(至少二年)。請詳見計畫書的規範、流程。
\end{outline}

}

\begin{columns}

\column{0.7}
\block{參加者資格}{
\begin{outline}
%\1 參加者資格
    \1 若具有以下資格,歡迎您加入:
        \2 口腔疣狀增生(oral verrucous hyperplasia, OVH): 經病理切片診斷確立
        \2 口腔白斑(oral leukoplakia): 經臨床診斷,僅收集血液、唾液;但日後若須要手術切片,剩餘病理檢體亦列入分析研究
    \1 而下列情況者,比較不適合參加本計劃:
        \2 病理切片手術後,診斷為口腔癌(oral squamous cell carcinoma, OSCC)
        \2 曾經罹患任何頭頸癌(例如甲狀腺癌、喉癌、鼻咽癌等)
        \2 (曾)接受任何癌症治療(化學治療、標靶治療、免疫治療、細胞治療等)
        \2 (曾)接受免疫調節藥物(自體免疫疾病、接受器官移植)
        \2 意識不清或心神喪失
\end{outline}
}

\column{0.3}
\block{研究預期之效益}{
%\begin{itemize}
這是一個前瞻性世代研究(prospective-longitudinal cohort study),接下來您會接受每三個月一次的常規醫療追蹤,我們也會幫助您戒除菸酒檳榔,萬一觀察到可疑病灶,即刻安排接受治療,以防止口腔癌的發生。藉由本計劃獲得的資訊,未來將有助於找出預測口腔癌化的生物標記(biomarker),以達到提早預防的目的。
%\end{itemize}
}

\end{columns}

\block{計畫主持人:吳家佑醫師\\
    臺北醫學大學附設醫院口腔顎面外科(臺北市信義區吳興街252號)\\
    電話:02-27372181 \#3211-7}{%
\begin{tikzfigure}[Figures in tikzposter]
\includegraphics[width=\linewidth]{istockphoto-622282400-612x612.jpg}
\end{tikzfigure}
}

\end{document}